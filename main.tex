% Template for ICASSP-2021 paper; to be used with:
%          spconf.sty  - ICASSP/ICIP LaTeX style file, and
%          IEEEbib.bst - IEEE bibliography style file.
% --------------------------------------------------------------------------
\documentclass{article}
\usepackage{spconf,amsmath,graphicx}

% Example definitions.
% --------------------
\def\x{{\mathbf x}}
\def\L{{\cal L}}

% Title.
% ------
\title{Few Shot Ear: Quick Learning of Percussive Sounds}
%
% Single address.
% ---------------
\name{Author(s) Name(s)\thanks{Thanks to XYZ agency for funding.}}
\address{Author Affiliation(s)}
%
% For example:
% ------------
%\address{School\\
%	Department\\
%	Address}
%
% Two addresses (uncomment and modify for two-address case).
% ----------------------------------------------------------
%\twoauthors
%  {A. Author-one, B. Author-two\sthanks{Thanks to XYZ agency for funding.}}
%	{School A-B\\
%	Department A-B\\
%	Address A-B}
%  {C. Author-three, D. Author-four\sthanks{The fourth author performed the work
%	while at ...}}
%	{School C-D\\
%	Department C-D\\
%	Address C-D}
%
\begin{document}
%\ninept
%
\maketitle
%
\begin{abstract}

We present an approach for learning of percussive sound categories with a small number of examples. 
A number of recent works have demonstrated effective learning and recreation of sounds given an adequate number of examples. 
However, effective learning given a small number of exemplars remains a difficult task.
Here, we utilize recent advancements in few-shot learning to train a virtual ear which can categorize drum types.
We demonstrate the accuracy of this virtual ear compared to traditional machine learning models when addressing the problem of ``open set recognition''.
Furthermore, we use this virtual ear to guide the synthesis of novel drum sounds using digital signal processing and demonstrate its
it's agreeableness with human surveyors. 
Few-shot learning of sounds is an important step towards recognition of open-sets of sounds and enables a large number of synthesis possibilities. We share our techniques and our datasets of sounds for replication and extension of this work. 

\end{abstract}
%
\begin{keywords}

\end{keywords}
%
\section{Introduction}
\label{sec:intro}

\bibliographystyle{IEEEbib}
\bibliography{refs}

\end{document}
