% Template for ICASSP-2021 paper; to be used with:
%          spconf.sty  - ICASSP/ICIP LaTeX style file, and
%          IEEEbib.bst - IEEE bibliography style file.
% --------------------------------------------------------------------------
\documentclass{article}
\usepackage{spconf,amsmath,graphicx}

% Example definitions.
% --------------------
\def\x{{\mathbf x}}
\def\L{{\cal L}}

% Title.
% ------
\title{Few Shot Ear: Quick Learning of Percussive Sounds}
%
% Single address.
% ---------------
\name{Amir Salimi and Abram Hindle
\address{University of Alberta\\
        Department of Computing Science\\
        % asalimi@ualberta.ca\\
        % hindle1@ualberta.ca\\
        }}

% For example:
% ------------
%\address{School\\
%	Department\\
%	Address}
%
% Two addresses (uncomment and modify for two-address case).
% ----------------------------------------------------------
%\twoauthors
%  {A. Author-one, B. Author-two\sthanks{Thanks to XYZ agency for funding.}}
%	{School A-B\\
%	Department A-B\\
%	Address A-B}
%  {C. Author-three, D. Author-four\sthanks{The fourth author performed the work
%	while at ...}}
%	{School C-D\\
%	Department C-D\\
%	Address C-D}
%
\begin{document}
%\ninept
%
\maketitle
%
\begin{abstract}
We present a methodology for machine learning of sound categories with a small number of examples. 
Given an adequate number of examples, automatic learning and recreation of sounds has been demonstrated in a number of recent works.
However, learning with a small number of exemplars remains a difficult task. Particularly troublesome is the recognition of open sets, often manifesting when the separation of data from noise is needed.
We utilize few-shot learning to train a virtual ear which can categorize drum types and guide the synthesis of novel drum sounds using a virtual sound synthesizer.
The success of this virtual ear in categorization of drum sounds and in the separation of noise from desirable sounds is shown in comparison to traditional approaches.
The Models and dataset of sounds needed for the replication and extension of this work are provided. 
\end{abstract}
%
\begin{keywords}

\end{keywords}
%
\section{Introduction}
\label{sec:intro}

\bibliographystyle{IEEEbib}
\bibliography{refs}

\end{document}
